\PassOptionsToPackage{unicode=true}{hyperref} % options for packages loaded elsewhere
\PassOptionsToPackage{hyphens}{url}
%
\documentclass[
  ignorenonframetext,
]{beamer}
\usepackage{pgfpages}
\setbeamertemplate{caption}[numbered]
\setbeamertemplate{caption label separator}{: }
\setbeamercolor{caption name}{fg=normal text.fg}
\beamertemplatenavigationsymbolsempty
% Prevent slide breaks in the middle of a paragraph:
\widowpenalties 1 10000
\raggedbottom
\setbeamertemplate{part page}{
  \centering
  \begin{beamercolorbox}[sep=16pt,center]{part title}
    \usebeamerfont{part title}\insertpart\par
  \end{beamercolorbox}
}
\setbeamertemplate{section page}{
  \centering
  \begin{beamercolorbox}[sep=12pt,center]{part title}
    \usebeamerfont{section title}\insertsection\par
  \end{beamercolorbox}
}
\setbeamertemplate{subsection page}{
  \centering
  \begin{beamercolorbox}[sep=8pt,center]{part title}
    \usebeamerfont{subsection title}\insertsubsection\par
  \end{beamercolorbox}
}
\AtBeginPart{
  \frame{\partpage}
}
\AtBeginSection{
  \ifbibliography
  \else
    \frame{\sectionpage}
  \fi
}
\AtBeginSubsection{
  \frame{\subsectionpage}
}
\usepackage{lmodern}
\usepackage{amssymb,amsmath}
\usepackage{ifxetex,ifluatex}
\ifnum 0\ifxetex 1\fi\ifluatex 1\fi=0 % if pdftex
  \usepackage[T1]{fontenc}
  \usepackage[utf8]{inputenc}
  \usepackage{textcomp} % provides euro and other symbols
\else % if luatex or xelatex
  \usepackage{unicode-math}
  \defaultfontfeatures{Scale=MatchLowercase}
  \defaultfontfeatures[\rmfamily]{Ligatures=TeX,Scale=1}
\fi
\usetheme[]{Boadilla}
\usecolortheme{dolphin}
\usefonttheme{professionalfonts}
% use upquote if available, for straight quotes in verbatim environments
\IfFileExists{upquote.sty}{\usepackage{upquote}}{}
\IfFileExists{microtype.sty}{% use microtype if available
  \usepackage[]{microtype}
  \UseMicrotypeSet[protrusion]{basicmath} % disable protrusion for tt fonts
}{}
\makeatletter
\@ifundefined{KOMAClassName}{% if non-KOMA class
  \IfFileExists{parskip.sty}{%
    \usepackage{parskip}
  }{% else
    \setlength{\parindent}{0pt}
    \setlength{\parskip}{6pt plus 2pt minus 1pt}}
}{% if KOMA class
  \KOMAoptions{parskip=half}}
\makeatother
\usepackage{xcolor}
\IfFileExists{xurl.sty}{\usepackage{xurl}}{} % add URL line breaks if available
\IfFileExists{bookmark.sty}{\usepackage{bookmark}}{\usepackage{hyperref}}
\hypersetup{
  pdftitle={Slide Template with Guidelines and Examples},
  pdfauthor={Mr.~Li; AP Research; Period 1},
  pdfborder={0 0 0},
  breaklinks=true}
\urlstyle{same}  % don't use monospace font for urls
\newif\ifbibliography
\setlength{\emergencystretch}{3em}  % prevent overfull lines
\providecommand{\tightlist}{%
  \setlength{\itemsep}{0pt}\setlength{\parskip}{0pt}}
\setcounter{secnumdepth}{-2}

% set default figure placement to htbp
\makeatletter
\def\fps@figure{htbp}
\makeatother

%need for R stargazer output
\usepackage{dcolumn}

\usepackage{caption} 
\captionsetup[table]{position=top, skip=-0.05in}

\title{Slide Template with Guidelines and Examples}
\author{Mr.~Li \and AP Research \and Period 1}
\date{2019-06-11}

\begin{document}
\frame{\titlepage}

\begin{frame}{Introduction}
\protect\hypertarget{introduction}{}

Introduce research topic and question.

What is the relevance of the question?

\end{frame}

\begin{frame}{Literature Review}
\protect\hypertarget{literature-review}{}

Theory A (Aqda, Hamidi, \& Rahimi,
\protect\hyperlink{ref-Aqda11}{2011}).

Aqda et al. (\protect\hyperlink{ref-Aqda11}{2011}) also posited
Hypothesis D with preliminary Results B.

Theory B with Results X, Y, and Z (Angrist \& Lavy,
\protect\hyperlink{ref-Angrist02}{2002}).

\end{frame}

\begin{frame}{Hypothesis}
\protect\hypertarget{hypothesis}{}

Formulate hypothesis or set of hypotheses based on preliminary research.

How does the hypothesis connect with existing research? Integrate with
literature review.

\end{frame}

\begin{frame}{Research Design}
\protect\hypertarget{research-design}{}

How will you test your hypothesis?

What research design will you employ? Based on your chosen research
design, what relevant research methods will you apply?

\end{frame}

\begin{frame}{Methods}
\protect\hypertarget{methods}{}

Explain your choice of methods based on research design.

How will your methods test your hypothesis?

\end{frame}

\begin{frame}{Results}
\protect\hypertarget{results}{}

\begin{table}[!htbp] \centering 
  \caption{Wage Regression Models} 
  \label{} 
\small 
\begin{tabular}{@{\extracolsep{-10pt}}lD{.}{.}{-3} D{.}{.}{-3} D{.}{.}{-3} } 
\\[-1.8ex]\hline 
\hline \\[-1.8ex] 
 & \multicolumn{3}{c}{\textit{Dependent variable:}} \\ 
\cline{2-4} 
\\[-1.8ex] & \multicolumn{3}{c}{wage} \\ 
\\[-1.8ex] & \multicolumn{1}{c}{(1)} & \multicolumn{1}{c}{(2)} & \multicolumn{1}{c}{(3)}\\ 
\hline \\[-1.8ex] 
 tenure & 0.177^{***} & 0.198^{***} & 0.178^{***} \\ 
  & (0.021) & (0.024) & (0.021) \\ 
  exper &  & -0.022^{*} &  \\ 
  &  & (0.013) &  \\ 
  nonwhite &  &  & -0.517 \\ 
  &  &  & (0.498) \\ 
  Constant & 4.991^{***} & 5.258^{***} & 5.043^{***} \\ 
  & (0.185) & (0.243) & (0.192) \\ 
 \hline \\[-1.8ex] 
Observations & \multicolumn{1}{c}{526} & \multicolumn{1}{c}{526} & \multicolumn{1}{c}{526} \\ 
R$^{2}$ & \multicolumn{1}{c}{0.120} & \multicolumn{1}{c}{0.125} & \multicolumn{1}{c}{0.122} \\ 
Adjusted R$^{2}$ & \multicolumn{1}{c}{0.119} & \multicolumn{1}{c}{0.122} & \multicolumn{1}{c}{0.119} \\ 
\hline 
\hline \\[-1.8ex] 
\textit{Note:}  & \multicolumn{3}{r}{$^{*}$p$<$0.1; $^{**}$p$<$0.05; $^{***}$p$<$0.01} \\ 
\end{tabular} 
\end{table}

\end{frame}

\begin{frame}[fragile]{Analysis}
\protect\hypertarget{analysis}{}

Model 1: \(\widehat{\text{wage}} = 4.99 + 0.18~\text{tenure}\)

Model 2:
\(\widehat{\text{wage}} = 5.26 + 0.2~\text{tenure} -0.02 ~\text{exper}\)

Model 3:
\(\widehat{\text{wage}} = 5.04 + 0.18~\text{tenure} -0.52 ~\text{nonwhite}\)

Despite the coefficient of \texttt{exper} being statistically
significant (\(p =\) 0.089) at the 10\% level, the negative sign implies
that each year of experience is associated with a decrease in 2 cents in
average hourly earnings, holding all else constant. This
counterintuitive result is most likely due to multicollinearity, as the
correlation between \texttt{tenure} and \texttt{exper} is rather high
(\(r =\) 0.499).

\end{frame}

\begin{frame}{Evaluation}
\protect\hypertarget{evaluation}{}

How does your collected evidence address your hypothesis?

In light of the evidence, how might you qualify the statements contained
in your hypothesis?

What are the implications of your study?

\end{frame}

\begin{frame}{Conclusion}
\protect\hypertarget{conclusion}{}

Potential directions for future studies

\end{frame}

\begin{frame}{References}
\protect\hypertarget{references}{}

\hypertarget{refs}{}
\leavevmode\hypertarget{ref-Angrist02}{}%
Angrist, J., \& Lavy, V. (2002). New Evidence on Classroom Computers and
Pupil Learning. \emph{The Economic Journal}, \emph{112}(482), 735--765.
\url{https://doi.org/10.1111/1468-0297.00068}

\leavevmode\hypertarget{ref-Aqda11}{}%
Aqda, M. F., Hamidi, F., \& Rahimi, M. (2011). The comparative effect of
computer-aided instruction and traditional teaching on student's
creativity in math classes. \emph{Procedia Computer Science}, \emph{3},
266--270. \url{https://doi.org/10.1016/j.procs.2010.12.045}

\end{frame}

\printbibliography

\end{document}
