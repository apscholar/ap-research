\documentclass[]{article}
\usepackage{lmodern}
\usepackage{amssymb,amsmath}
\usepackage{ifxetex,ifluatex}
\usepackage{fixltx2e} % provides \textsubscript
\ifnum 0\ifxetex 1\fi\ifluatex 1\fi=0 % if pdftex
  \usepackage[T1]{fontenc}
  \usepackage[utf8]{inputenc}
\else % if luatex or xelatex
  \ifxetex
    \usepackage{mathspec}
  \else
    \usepackage{fontspec}
  \fi
  \defaultfontfeatures{Ligatures=TeX,Scale=MatchLowercase}
\fi
% use upquote if available, for straight quotes in verbatim environments
\IfFileExists{upquote.sty}{\usepackage{upquote}}{}
% use microtype if available
\IfFileExists{microtype.sty}{%
\usepackage{microtype}
\UseMicrotypeSet[protrusion]{basicmath} % disable protrusion for tt fonts
}{}
\usepackage[margin=1in]{geometry}
\usepackage{hyperref}
\hypersetup{unicode=true,
            pdftitle={Oral Defense Questions},
            pdfborder={0 0 0},
            breaklinks=true}
\urlstyle{same}  % don't use monospace font for urls
\usepackage{graphicx,grffile}
\makeatletter
\def\maxwidth{\ifdim\Gin@nat@width>\linewidth\linewidth\else\Gin@nat@width\fi}
\def\maxheight{\ifdim\Gin@nat@height>\textheight\textheight\else\Gin@nat@height\fi}
\makeatother
% Scale images if necessary, so that they will not overflow the page
% margins by default, and it is still possible to overwrite the defaults
% using explicit options in \includegraphics[width, height, ...]{}
\setkeys{Gin}{width=\maxwidth,height=\maxheight,keepaspectratio}
\IfFileExists{parskip.sty}{%
\usepackage{parskip}
}{% else
\setlength{\parindent}{0pt}
\setlength{\parskip}{6pt plus 2pt minus 1pt}
}
\setlength{\emergencystretch}{3em}  % prevent overfull lines
\providecommand{\tightlist}{%
  \setlength{\itemsep}{0pt}\setlength{\parskip}{0pt}}
\setcounter{secnumdepth}{0}
% Redefines (sub)paragraphs to behave more like sections
\ifx\paragraph\undefined\else
\let\oldparagraph\paragraph
\renewcommand{\paragraph}[1]{\oldparagraph{#1}\mbox{}}
\fi
\ifx\subparagraph\undefined\else
\let\oldsubparagraph\subparagraph
\renewcommand{\subparagraph}[1]{\oldsubparagraph{#1}\mbox{}}
\fi

%%% Use protect on footnotes to avoid problems with footnotes in titles
\let\rmarkdownfootnote\footnote%
\def\footnote{\protect\rmarkdownfootnote}

%%% Change title format to be more compact
\usepackage{titling}

% Create subtitle command for use in maketitle
\providecommand{\subtitle}[1]{
  \posttitle{
    \begin{center}\large#1\end{center}
    }
}

\setlength{\droptitle}{-2em}

  \title{Oral Defense Questions}
    \pretitle{\vspace{\droptitle}\centering\huge}
  \posttitle{\par}
    \author{\emph{AP Research}}
    \preauthor{\centering\large\emph}
  \postauthor{\par}
      \predate{\centering\large\emph}
  \postdate{\par}
    \date{\emph{2019-07-21}}


\begin{document}
\maketitle

You can find details pertaining to the oral defense in pages 52--53 of
the
\href{https://apcentral.collegeboard.org/pdf/ap-research-course-and-exam-description.pdf?course=ap-research}{AP
Research Course and Exam Description}. The questions are reproduced
below so you can fill out responses in preparation for the oral defense.
During the oral defense, you will be asked one question in each of the
three sections below.

\hypertarget{researchinquiry-process}{%
\section{Research/Inquiry Process}\label{researchinquiry-process}}

\begin{itemize}
\item
  After you chose your research question/project goal, which information
  guided your choice of a research method/artistic process?
\item
  How is the method/process you chose aligned with the purpose of your
  research? Which methods did you consider and reject?
\item
  What were the strategies you used to conduct a review of the
  literature or gather information from the discipline-specific field?
  Why did you select those strategies? Which strategies did you consider
  and reject?
\item
  How did you evaluate the sources you collected to make sure they would
  be credible,valid, and reliable? Which sources did you discard, and
  why?
\item
  What was one obstacle or challenge you encountered while implementing
  yourresearch method, and how did you address it?
\item
  What was the most important source of information you found while
  conducting your research, and why was it important to your research
  process?
\end{itemize}

\hypertarget{depth-of-understanding}{%
\section{Depth of Understanding}\label{depth-of-understanding}}

\begin{itemize}
\item
  What was the fundamental argument/idea in your research? How does this
  argument/idea relate to the primary purpose of your research?
\item
  Which of the various perspectives you explored was the most difficult
  for you to incorporate into your research inquiry, and why?
\item
  What criteria did you use to discriminate among the perspectives in
  order to reach a conclusion?
\item
  How might your conclusions/findings/product relate(s) to the current
  body of work in the community or field?
\item
  What might be the real-world implications or consequences (influence
  on others' behaviors, decision-making processes, or discoveries)
  related to your findings?
\item
  What additional questions emerged during your research? Based on your
  recent experience, what advice would you give to other researchers who
  might choose to investigate those questions?
\end{itemize}

\hypertarget{reflection-throughout-the-inquiry-process}{%
\section{Reflection Throughout the Inquiry
Process}\label{reflection-throughout-the-inquiry-process}}

\begin{itemize}
\item
  Which of your sources was the most influential, and in what way is
  that influence apparent in your final conclusion or result?
\item
  In which specific part of your research process was your expert
  adviser most helpful,and how was he or she most helpful? What did you
  learn from the expert adviser about your field of research? Note: do
  not ask this question if the student did not engage with an expert
  adviser.
\item
  If you could revisit the research process, what would you do
  differently? Would you choose a different area of inquiry, and if so,
  why? If you would choose the same research question/project goal, what
  different methods or approaches would you use?
\item
  If you had three more months to work on this research question/project
  goal, what additional research strategies would you put into practice?
\item
  Think about the initial curiosity that led to your inquiry. What other
  areas of inquiry might that same curiosity lead to?
\item
  What unanticipated turn did you encounter as your research progressed?
  What were the reasons for this change in direction or focus, and how
  did you modify your method or approach?
\end{itemize}


\end{document}
